\documentclass[OPS,lsstdraft,authoryear,toc]{lsstdoc}
\input{meta}

% Package imports go here.

% Local commands go here.

%If you want glossaries
%\input{aglossary.tex}
%\makeglossaries

\title{Rubin Observatory Raw Data File Format}

% This can write metadata into the PDF.
% Update keywords and author information as necessary.
\hypersetup{
    pdftitle={Rubin Observatory Raw Data File Format},
    pdfauthor={TimJenness},
    pdfkeywords={}
}

% Optional subtitle
% \setDocSubtitle{A subtitle}

\input{authors}

\setDocRef{CTN-004}
\setDocUpstreamLocation{\url{https://github.com/lsst/ctn-004}}

\date{\vcsDate}

% Optional: name of the document's curator
% \setDocCurator{The Curator of this Document}

\setDocAbstract{%
At the NSF-DOE Vera C. Rubin Observatory we write data from the LSSTCam and LATISS instruments using FITS format with one file written per detector. Here we discuss the layout of those FITS files and describe the FITS headers.
}

% Change history defined here.
% Order: oldest first.
% Fields: VERSION, DATE, DESCRIPTION, OWNER NAME.
% See LPM-51 for version number policy.
\setDocChangeRecord{%
  \addtohist{1}{YYYY-MM-DD}{Unreleased.}{Jenness}
}


\begin{document}

% Create the title page.
\maketitle
% Frequently for a technote we do not want a title page  uncomment this to remove the title page and changelog.
% use \mkshorttitle to remove the extra pages

% ADD CONTENT HERE
% You can also use the \input command to include several content files.

\section{Introduction}

All three cameras at the NSF-DOE Vera C.\ Rubin Observatory, (LSSTCam \citep{10.71929/rubin/2571927}, LATISS \citep{10.71929/rubin/2571930}, and LSSTComCam \citep{10.71929/rubin/2561361}) write FITS output files in the same structure with some minor variations in the FITS headers indicating different software versions and capabilities.

\citeds{LCA-10140} and \citeds{LCA-13501} define the file format used for the test stand at SLAC and \citeds{LSE-400} describes an example header that was used early in the construction project.
This document describes the delivered system and any evolution of the format that happens during the observatory lifetime.

\section{Structure}

The basic layout of a Rubin camera FITS file is shown in Fig.~\ref{fig:layout}.
There is a primary header, with no data, containing the FITS headers for this observation.
Then there are 16 HDUs, named \texttt{SegmentNN}, with the data from each amplifier along with headers specific to that amplifier.
The data values are 32-bit integers compressed using the Rice compression algorithm \citep{2012arXiv1201.1336W}.
Finally there are two extensions describing the condition of the readout electronics board (\texttt{REB\_COND}) and checksums calculated from the camera configuration (\texttt{CONFIG\_COND}).

\begin{figure}
  \begin{center}
\begin{verbatim}
No.    Name      Ver    Type      Cards   Dimensions   Format
  0  PRIMARY       1 PrimaryHDU     134   ()
  1  Segment10     1 CompImageHDU    115   (576, 2048)   int32
  2  Segment11     1 CompImageHDU    115   (576, 2048)   int32
  3  Segment12     1 CompImageHDU    115   (576, 2048)   int32
  4  Segment13     1 CompImageHDU    115   (576, 2048)   int32
  5  Segment14     1 CompImageHDU    115   (576, 2048)   int32
  6  Segment15     1 CompImageHDU    115   (576, 2048)   int32
  7  Segment16     1 CompImageHDU    115   (576, 2048)   int32
  8  Segment17     1 CompImageHDU    115   (576, 2048)   int32
  9  Segment07     1 CompImageHDU    115   (576, 2048)   int32
 10  Segment06     1 CompImageHDU    115   (576, 2048)   int32
 11  Segment05     1 CompImageHDU    115   (576, 2048)   int32
 12  Segment04     1 CompImageHDU    115   (576, 2048)   int32
 13  Segment03     1 CompImageHDU    115   (576, 2048)   int32
 14  Segment02     1 CompImageHDU    115   (576, 2048)   int32
 15  Segment01     1 CompImageHDU    115   (576, 2048)   int32
 16  Segment00     1 CompImageHDU    115   (576, 2048)   int32
 17  REB_COND      1 BinTableHDU    107   0R x 0C   []
 18  CONFIG_COND    1 BinTableHDU     26   0R x 0C   []
\end{verbatim}
\end{center}
\caption{Layout of a LSSTCam raw file.}
\label{fig:layout}
\end{figure}

\section{FITS Headers}

\subsection{LSSTCam Header}

The definitions of how specific timing FITS headers relate to events inside the camera are described in \citeds{CTN-005}.


\subsubsection{Date, night and basic image information}


\begin{tabular}{l l l l l}
\hline
Header & Type & Description \\
\hline
DATE & Date & Creation Date and Time of File \\
MJD & MJD & Modified Julian Date that the file was written \\
IMGTYPE & String & BIAS, DARK, FE55, FLAT, FLAT\_<lam>, SPOT, PPUMP \\
DATE-OBS & Date & Time at the start of integration \\
MJD-OBS & MJD & Modified Julian Date derived from DATE-OBS \\
DATE-TRG & Date & TAI Date of the image trigger (readout) \\
MJD-TRG & MJD & Modified Julian Date of image trigger \\
OBSID & String & The image name or obs-id \\
DATE-BEG & Date & Time at the start of integration \\
MJD-BEG & MJD & Modified Julian Date derived from DATE-BEG \\
DATE-END & Date & End date of the observation \\
MJD-END & MJD & Date derived from DATE-END \\
BUNIT & String & Brightness units for pixel array \\
TIMESYS & String & The time scale used \\
GROUPID & String &  \\
\hline
\end{tabular}


\subsubsection{Telescope info, location, observer}


\begin{tabular}{l l l l l}
\hline
Header & Type & Description \\
\hline
INSTRUME & String & Instrument \\
TELESCOP & String & Telescope \\
OBS-LONG & String & [deg] Observatory east longitude \\
OBS-LAT & String & [deg] Observatory latitude \\
OBS-ELEV & String & [m] Observatory elevation \\
OBSGEO-X & String & [m] X-axis Geocentric coordinate \\
OBSGEO-Y & String & [m] Y-axis Geocentric coordinate \\
OBSGEO-Z & String & [m] Z-axis Geocentric coordinate \\
\hline
\end{tabular}


\subsubsection{Pointing info, etc.}


\begin{tabular}{l l l l l}
\hline
Header & Type & Description \\
\hline
RA & String & RA commanded from pointing component \\
DEC & String & DEC commanded from pointing component \\
RASTART & String & RA of telescope from AZSTART and ELSTART \\
DECSTART & String & DEC of telescope from AZSTART and ELSTART \\
RAEND & String & RA of telescope from AZEND and ELEND \\
DECEND & String & DEC of telescope from AZEND and ELEND \\
ROTPA & String & Rotation angle relative to the sky (deg) \\
ROTCOORD & String & Telescope Rotation Coordinates \\
HASTART & String & [HH:MM:SS] Telescope hour angle at start \\
ELSTART & String & [deg] Telescope zenith distance at start \\
AZSTART & String & [deg] Telescope azimuth angle at start \\
AMSTART & String & Airmass at start \\
HAEND & String & [HH:MM:SS] Telescope hour angle at end \\
ELEND & String & [deg] Telescope zenith distance at end \\
AZEND & String & [deg] Telescope azimuth angle at end \\
AMEND & String & Airmass at end \\
TRACKSYS & String & Tracking system RADEC, AZEL, PLANET, EPHEM \\
RADESYS & String & Equatorial coordinate system FK5 or ICRS \\
FOCUSZ & String & Focus Z position \\
OBJECT & String & Name of the observed object \\
VIGNETTE & String & Is the telescope vignetted? \\
VIGN\_MIN & String & Minimum vignetting? \\
\hline
\end{tabular}


\subsubsection{Image-identifying used to build OBS-ID}


\begin{tabular}{l l l l l}
\hline
Header & Type & Description \\
\hline
TESTTYPE & String & BIAS, DARK, FE55, FLAT, LAMBDA, PERSISTENCE, SPOT, SFLAT\_<lam>, TRAP \\
CAMCODE & String & The "code" for AuxTel | ComCam | Main Camera \\
CONTRLLR & String & The controller (e.g. O for OCS, C for CCS) \\
DAYOBS & String & The observation day as defined in the image name  \\
SEQNUM & Integer & The sequence number from the image name \\
EMUIMAGE & String & Image being emulated (from 2-day store) \\
\hline
\end{tabular}


\subsubsection{Additional Keys Information from Camera}


\begin{tabular}{l l l l l}
\hline
Header & Type & Description \\
\hline
PROGRAM & String & Name of the program \\
REASON & String & Reason for observation \\
\hline
\end{tabular}


\subsubsection{Image sequence numbers}


\begin{tabular}{l l l l l}
\hline
Header & Type & Description \\
\hline
CURINDEX & Integer & Index number for exposure within the sequence \\
MAXINDEX & Integer & Number of requested images in sequence \\
\hline
\end{tabular}


\subsubsection{Test Stand information}


\begin{tabular}{l l l l l}
\hline
Header & Type & Description \\
\hline
TSTAND & String & Test Stand/Physical location \\
\hline
\end{tabular}


\subsubsection{Information from Camera (Common block)}


\begin{tabular}{l l l l l}
\hline
Header & Type & Description \\
\hline
IMAGETAG & String & DAQ Image id (Hex) \\
OBSANNOT & String & DAQ image annotation \\
\hline
\end{tabular}


\subsubsection{Information from Camera}


\begin{tabular}{l l l l l}
\hline
Header & Type & Description \\
\hline
RUNNUM & String & The Run Number \\
CONTROLL & String & Duplicates INSTRUME \\
CCD\_MANU & String & CCD Manufacturer \\
CCD\_TYPE & String & CCD Model Number \\
TEMP\_SET & Float & Temperature set point (deg C) \\
CCDSLOT & String & The CCD Slot \\
RAFTBAY & String & The RAFT Bay \\
FIRMWARE & String & DAQ firmware version (Hex) \\
PLATFORM & String & DAQ platform version \\
CONTNUM & String & REB serial \# (Hex) \\
DAQVERS & String & DAQ version \\
DAQPART & String & DAQ partition \\
DAQFOLD & String & DAQ folder the image was initially created in \\
SEQFILE & String & Sequencer file name \\
SEQNAME & String & Sequencer file name \\
SEQCKSUM & String & Checksum of Sequencer \\
LSST\_NUM & String & LSST Assigned CCD Number \\
CCD\_SERN & String & Manufacturers’ CCD Serial Number \\
REBNAME & String & LSST Assigned Name REB name \\
RAFTNAME & String & LSST Assigned Raft name \\
FPVERS & String & The focal-plane version number \\
IHVERS & String & The image-handling version number \\
\hline
\end{tabular}


\subsubsection{Filter/grating information}


\begin{tabular}{l l l l l}
\hline
Header & Type & Description \\
\hline
FILTBAND & String & Name of the filter band \\
FILTER & String & Name of the physical filter \\
FILTPOS & String & Filter measured position of filter \\
FILTSLOT & String & Filter home slot \\
\hline
\end{tabular}


\subsubsection{Exposure-related information}


\begin{tabular}{l l l l l}
\hline
Header & Type & Description \\
\hline
EXPTIME & Float & Exposure Time in Seconds \\
DARKTIME & Float & Dark Time in Seconds (see TSEIA-91) \\
SHUTTIME & String & Measured shutter exposure time \\
\hline
\end{tabular}


\subsubsection{Weather information}


\begin{tabular}{l l l l l}
\hline
Header & Type & Description \\
\hline
AIRTEMP & String & Position of slide \\
PRESSURE & String & [Pascals] Atmospheric Pressure \\
HUMIDITY & String & Percentage of Relative Humidity \\
WINDSPD & String & [m/s] Wind Speed \\
WINDDIR & String & [deg] Wind Direction \\
SEEING & String & [arcsec] Seeing from FWHM DIMM measurement \\
\hline
\end{tabular}


\subsubsection{Header information}


\begin{tabular}{l l l l l}
\hline
Header & Type & Description \\
\hline
HEADVER & Integer & Version number of header \\
FILENAME & String & Original name of the file \\
\hline
\end{tabular}


\subsubsection{Stuttered Image Support}


\begin{tabular}{l l l l l}
\hline
Header & Type & Description \\
\hline
HIERARCH.STUTTER.ROWS & Integer & The number of row shifts per stutter \\
HIERARCH.STUTTER.DELAY & Float & [s] The delay between stutters \\
HIERARCH.STUTTER.NSHIFTS & Integer & The number of stutters \\
\hline
\end{tabular}



\subsection{LATISS Header}


\subsubsection{Date, night and basic image information}


\begin{tabular}{l l l l l}
\hline
Header & Type & Description \\
\hline
DATE & Date & Creation Date and Time of File \\
MJD & MJD & Modified Julian Date that the file was written \\
IMGTYPE & String & BIAS, DARK, FE55, FLAT, FLAT\_<lam>, SPOT, PPUMP \\
DATE-OBS & Date & Time at the start of integration \\
MJD-OBS & MJD & Modified Julian Date derived from DATE-OBS \\
DATE-TRG & Date & TAI Date of the image trigger (readout) \\
MJD-TRG & MJD & Modified Julian Date of image trigger \\
OBSID & String & The image name or obs-id \\
DATE-BEG & Date & Time at the start of integration \\
MJD-BEG & MJD & Modified Julian Date derived from DATE-BEG \\
DATE-END & Date & End date of the observation \\
MJD-END & MJD & Date derived from DATE-END \\
BUNIT & String & Brightness units for pixel array \\
TIMESYS & String & The time scale used \\
GROUPID & String &  \\
\hline
\end{tabular}


\subsubsection{Telescope info, location, observer}


\begin{tabular}{l l l l l}
\hline
Header & Type & Description \\
\hline
INSTRUME & String & Instrument \\
TELESCOP & String & Telescope \\
OBS-LONG & String & [deg] Observatory east longitude \\
OBS-LAT & String & [deg] Observatory latitude \\
OBS-ELEV & String & [m] Observatory elevation \\
OBSGEO-X & String & [m] X-axis Geocentric coordinate \\
OBSGEO-Y & String & [m] Y-axis Geocentric coordinate \\
OBSGEO-Z & String & [m] Z-axis Geocentric coordinate \\
FACILITY & String & Facility name \\
\hline
\end{tabular}


\subsubsection{Pointing info, etc.}


\begin{tabular}{l l l l l}
\hline
Header & Type & Description \\
\hline
RA & String & RA commanded from pointing component \\
DEC & String & DEC commanded from pointing component \\
RASTART & String & RA of telescope from AZSTART and ELSTART \\
DECSTART & String & DEC of telescope from AZSTART and ELSTART \\
RAEND & String & RA of telescope from AZEND and ELEND \\
DECEND & String & DEC of telescope from AZEND and ELEND \\
ROTPA & String & Rotation angle relative to the sky (deg) \\
ROTCOORD & String & Telescope Rotation Coordinates \\
HASTART & String & [HH:MM:SS] Telescope hour angle at start \\
ELSTART & String & [deg] Telescope zenith distance at start \\
AZSTART & String & [deg] Telescope azimuth angle at start \\
AMSTART & String & Airmass at start \\
HAEND & String & [HH:MM:SS] Telescope hour angle at end \\
ELEND & String & [deg] Telescope zenith distance at end \\
AZEND & String & [deg] Telescope azimuth angle at end \\
AMEND & String & Airmass at end \\
TRACKSYS & String & Tracking system RADEC, AZEL, PLANET, EPHEM \\
RADESYS & String & Equatorial coordinate system FK5 or ICRS \\
FOCUSZ & String & Focus Z position \\
OBJECT & String & Name of the observed object \\
VIGNETTE & String & Is the telescope vignetted? \\
VIGN\_MIN & String & Minimum vignetting? \\
SHUTLOWR & String & Dome Dropout Door Opening Percentage \\
SHUTUPPR & String & Dome Main Door Opening Percentage \\
DOMEAZ & String & [deg] Dome Azimuth Position \\
INSTPORT & String & The instrument port where LATISS is installed \\
ATM3PORT & String & The port the M3 is directing light into \\
\hline
\end{tabular}


\subsubsection{TAN Projection}


\begin{tabular}{l l l l l}
\hline
Header & Type & Description \\
\hline
WCSAXES & String & WCS Dimensionality \\
CTYPE1 & String & Coordinate type \\
CTYPE2 & String & Coordinate type \\
CUNIT1 & String &  \\
CUNIT2 & String &  \\
CRVAL1 & String & [deg] WCS Reference Coordinate (RA) \\
CRVAL2 & String & [deg] WCS Reference Coordinate (DEC) \\
CRPIX1 & String & Reference pixel axis 1 \\
CRPIX2 & String & Reference pixel axis 2 \\
CD1\_1 & String & DL/DX World coordinate transformation matrix \\
CD1\_2 & String & DL/DY World coordinate transformation matrix \\
CD2\_1 & String & DM/DX World coordinate transformation matrix \\
CD2\_2 & String & DM/DY World coordinate transformation matrix \\
EQUINOX & String & Equinox of coordinates \\
\hline
\end{tabular}


\subsubsection{Image-identifying used to build OBS-ID}


\begin{tabular}{l l l l l}
\hline
Header & Type & Description \\
\hline
TESTTYPE & String & BIAS, DARK, FE55, FLAT, LAMBDA, PERSISTENCE, SPOT, SFLAT\_<lam>, TRAP \\
CAMCODE & String & The "code" for AuxTel | ComCam | Main Camera \\
CONTRLLR & String & The controller (e.g. O for OCS, C for CCS) \\
DAYOBS & String & The observation day as defined in the image name  \\
SEQNUM & Integer & The sequence number from the image name \\
EMUIMAGE & String & Image being emulated (from 2-day store) \\
\hline
\end{tabular}


\subsubsection{Additional Keys Information from Camera}


\begin{tabular}{l l l l l}
\hline
Header & Type & Description \\
\hline
PROGRAM & String & Name of the program \\
REASON & String & Reason for observation \\
\hline
\end{tabular}


\subsubsection{Image sequence numbers}


\begin{tabular}{l l l l l}
\hline
Header & Type & Description \\
\hline
CURINDEX & Integer & Index number for exposure within the sequence \\
MAXINDEX & Integer & Number of requested images in sequence \\
\hline
\end{tabular}


\subsubsection{Test Stand information}


\begin{tabular}{l l l l l}
\hline
Header & Type & Description \\
\hline
TSTAND & String & Camera test stand BOT or CCOB \\
\hline
\end{tabular}


\subsubsection{Information from Camera (Common block)}


\begin{tabular}{l l l l l}
\hline
Header & Type & Description \\
\hline
IMAGETAG & String & DAQ Image id (Hex) \\
OBSANNOT & String & DAQ image annotation \\
\hline
\end{tabular}


\subsubsection{Information from Camera}


\begin{tabular}{l l l l l}
\hline
Header & Type & Description \\
\hline
RUNNUM & String & The Run Number \\
CONTROLL & String & Duplicates INSTRUME \\
CCD\_MANU & String & CCD Manufacturer \\
CCD\_TYPE & String & CCD Model Number \\
TEMP\_SET & Float & CCD Temperature set point (deg C) \\
CCDSLOT & String & The CCD Slot \\
RAFTBAY & String & The RAFT Bay \\
FIRMWARE & String & DAQ firmware version (Hex) \\
PLATFORM & String & DAQ platform version \\
CONTNUM & String & REB serial \# (Hex) \\
DAQVERS & String & DAQ version \\
DAQPART & String & DAQ partition \\
DAQFOLD & String & DAQ folder the image was initially created in \\
SEQFILE & String & Sequencer file name \\
SEQNAME & String & Sequencer file name \\
SEQCKSUM & String & Checksum of Sequencer \\
LSST\_NUM & String & LSST Assigned CCD Number \\
CCD\_SERN & String & Manufacturers’ CCD Serial Number \\
REBNAME & String & LSST Assigned Name REB name \\
RAFTNAME & String & LSST Assigned Raft name \\
FPVERS & String & The focal-plane version number \\
IHVERS & String & The image-handling version number \\
\hline
\end{tabular}


\subsubsection{Filter/grating information}


\begin{tabular}{l l l l l}
\hline
Header & Type & Description \\
\hline
GRATING & String & Name of physical grating \\
GRATBAND & String & Descriptive name \\
GRATSLOT & String & Grating's slot (1-indexed) \\
LINSPOS & String & Position of slide \\
FILTBAND & String & Name of the filter band \\
FILTER & String & Name of the physical filter \\
FILTPOS & String & Filter measured position of slide \\
FILTSLOT & String & Filter home slot \\
\hline
\end{tabular}


\subsubsection{Exposure-related information}


\begin{tabular}{l l l l l}
\hline
Header & Type & Description \\
\hline
EXPTIME & Float & Exposure Time in Seconds \\
DARKTIME & Float & Dark Time in Seconds (see TSEIA-91) \\
SHUTTIME & String & Shutter exposure time \\
\hline
\end{tabular}


\subsubsection{Weather information}


\begin{tabular}{l l l l l}
\hline
Header & Type & Description \\
\hline
AIRTEMP & String & Position of slide \\
PRESSURE & String & [Pascals] Atmospheric Pressure \\
HUMIDITY & String & Percentage of Relative Humidity \\
WINDSPD & String & [m/s] Wind Speed \\
WINDDIR & String & [deg] Wind Direction \\
SEEING & String & [arcsec] Seeing from FWHM DIMM measurement \\
\hline
\end{tabular}


\subsubsection{Header information}


\begin{tabular}{l l l l l}
\hline
Header & Type & Description \\
\hline
HEADVER & Integer & Version number of header \\
FILENAME & String & Original name of the file \\
\hline
\end{tabular}


\subsubsection{Hierarch information for CSC Simulatiom Mode}


\begin{tabular}{l l l l l}
\hline
Header & Type & Description \\
\hline
HIERARCH.SIMULATE.ATMCS & String & ATMCS Simulation Mode (False=0) \\
HIERARCH.SIMULATE.ATHEXAPOD & String & ATHexapod Simulation Mode (False=0) \\
HIERARCH.SIMULATE.ATPNEUMATICS & String & ATPneumatics Simulation Mode (False=0) \\
HIERARCH.SIMULATE.ATDOME & String & ATDome Simulation Mode (False=0) \\
HIERARCH.SIMULATE.ATSPECTROGRAPH & String & ATSpectrograph Simulation Mode (False=0) \\
\hline
\end{tabular}


\subsubsection{Stuttered Image Support}


\begin{tabular}{l l l l l}
\hline
Header & Type & Description \\
\hline
HIERARCH.STUTTER.ROWS & Integer & The number of row shifts per stutter \\
HIERARCH.STUTTER.DELAY & Float & [s] The delay between stutters \\
HIERARCH.STUTTER.NSHIFTS & Integer & The number of stutters \\
\hline
\end{tabular}



\subsection{Amplifier Header}

Header Version: 2

\subsubsection{No Group Assigned}


\begin{tabular}{l l l l l}
\hline
Header & Type & Description \\
\hline
INHERIT & Boolean & Extension inherits values from primary header \\
NAXIS & Integer & number of axis \\
NAXIS1 & Integer & size of the n'th axis \\
NAXIS2 & Integer & size of the n'th axis \\
CHANNEL & Integer &  \\
EXTNAME & String &  \\
CCDSUM & String &  \\
AVERAGE & Float &  \\
AVGBIAS & Float &  \\
AVWOBIAS & Float &  \\
STDVBIAS & Float &  \\
STDEV & Float &  \\
DATASEC & String &  \\
DETSEC & String &  \\
DETSIZE & String &  \\
DTV1 & Integer & detector transformation vector \\
DTV2 & Integer & detector transformation vector \\
DTM1\_1 & Float & detector transformation matrix \\
DTM2\_2 & Float & detector transformation matrix \\
DTM1\_2 & Float & detector transformation matrix \\
DTM2\_1 & Float & detector transformation matrix \\
WCSNAMEA & String & Name of coordinate system \\
CTYPE1A & String & In the camera coordinate system \\
CTYPE2A & String & In the camera coordinate system \\
PC1\_1A & Float &  \\
PC1\_2A & Float &  \\
PC2\_1A & Float &  \\
PC2\_2A & Float &  \\
CDELT1A & Float &  \\
CDELT2A & Float &  \\
CRPIX1A & Float &  \\
CRPIX2A & Float &  \\
CRVAL1A & Float &  \\
CRVAL2A & Float &  \\
WCSNAMEC & String & Name of coordinate system \\
CTYPE1C & String & In the camera coordinate system \\
CTYPE2C & String & In the camera coordinate system \\
PC1\_1C & Float &  \\
PC1\_2C & Float &  \\
PC2\_1C & Float &  \\
PC2\_2C & Float &  \\
CDELT1C & Float &  \\
CDELT2C & Float &  \\
CRPIX1C & Float &  \\
CRPIX2C & Float &  \\
CRVAL1C & Float &  \\
CRVAL2C & Float &  \\
WCSNAMER & String & Name of coordinate system \\
CTYPE1R & String & In the camera coordinate system \\
CTYPE2R & String & In the camera coordinate system \\
PC1\_1R & Float &  \\
PC1\_2R & Float &  \\
PC2\_1R & Float &  \\
PC2\_2R & Float &  \\
CDELT1R & Float &  \\
CDELT2R & Float &  \\
CRPIX1R & Float &  \\
CRPIX2R & Float &  \\
CRVAL1R & Float &  \\
CRVAL2R & Float &  \\
WCSNAMEF & String & Name of coordinate system \\
CTYPE1F & String & In the camera coordinate system \\
CTYPE2F & String & In the camera coordinate system \\
PC1\_1F & Float &  \\
PC1\_2F & Float &  \\
PC2\_1F & Float &  \\
PC2\_2F & Float &  \\
CDELT1F & Float &  \\
CDELT2F & Float &  \\
CRPIX1F & Float &  \\
CRPIX2F & Float &  \\
CRVAL1F & Float &  \\
CRVAL2F & Float &  \\
WCSNAMEE & String & Name of coordinate system \\
CTYPE1E & String & In the camera coordinate system \\
CTYPE2E & String & In the camera coordinate system \\
PC1\_1E & Float &  \\
PC1\_2E & Float &  \\
PC2\_1E & Float &  \\
PC2\_2E & Float &  \\
CDELT1E & Float &  \\
CDELT2E & Float &  \\
CRPIX1E & Float &  \\
CRPIX2E & Float &  \\
CRVAL1E & Float &  \\
CRVAL2E & Float &  \\
WCSNAMEB & String & Name of coordinate system \\
CTYPE1B & String & In the serial-parallel coordinate system \\
CTYPE2B & String & In the serial-parallel coordinate system \\
PC1\_1B & Float &  \\
PC1\_2B & Float &  \\
PC2\_1B & Float &  \\
PC2\_2B & Float &  \\
CDELT1B & Float &  \\
CDELT2B & Float &  \\
CRPIX1B & Float &  \\
CRPIX2B & Float &  \\
CRVAL1B & Float &  \\
CRVAL2B & Float &  \\
WCSNAMEQ & String & Name of coordinate system \\
CTYPE1Q & String & In the serial-parallel coordinate system \\
CTYPE2Q & String & In the serial-parallel coordinate system \\
PC1\_1Q & Float &  \\
PC1\_2Q & Float &  \\
PC2\_1Q & Float &  \\
PC2\_2Q & Float &  \\
CDELT1Q & Float &  \\
CDELT2Q & Float &  \\
CRPIX1Q & Float &  \\
CRPIX2Q & Float &  \\
CRVAL1Q & Float &  \\
CRVAL2Q & Float &  \\
\hline
\end{tabular}



\subsection{REB Condition Header}

Header Version: 2


\begin{longtable}{l l l l l}

\hline
Header & Type & Description \\
\hline
EXTNAME & String & Name of the extension \\
TEMP1 & Float & REB Board temperature 1 \\
TEMP2 & Float & REB Board temperature 2 \\
TEMP3 & Float & REB Board temperature 3 \\
TEMP4 & Float & REB Board temperature 4 \\
TEMP5 & Float & REB Board temperature 5 \\
TEMP6 & Float & REB Board temperature 6 \\
TEMP7 & Float & REB Board temperature 7 \\
TEMP8 & Float & REB Board temperature 8 \\
TEMP9 & Float & REB Board temperature 9 \\
TEMP10 & Float & REB Board temperature 10 \\
ATEMPU & Float & ASPIC 0 upper temperature \\
ATEMPL & Float & ASPIC 0 lower temperature \\
CCDTEMP & Float & CCD temperature \\
RTDTEMP & Float & RTD temperature \\
TEMPAVG & Float & REB temp6/temp10 averaged during integration \\
DIGPS\_V & Float & REB Digital PS voltage  \\
DIGPS\_I & Float & REB Digital PS current  \\
ANAPS\_V & Float & REB Analog PS voltage  \\
ANAPS\_I & Float & REB Analog PS current  \\
CLKHPS\_V & Float & REB CLK\_H PS voltage  \\
CLKHPS\_I & Float & REB CLK\_H PS current  \\
CLKLPS\_V & Float & REB CLK\_L PS voltage  \\
CLKLPS\_I & Float & REB CLK\_L PS current  \\
ODPS\_V & Float & REB OD PS voltage  \\
ODPS\_I & Float & REB OD PS current  \\
HTRPS\_V & Float & REB Heater PS voltage  \\
HTRPS\_W & Float & REB Heater PS power  \\
PCKU\_V & Float & REB clock rails PCK\_U Voltage \\
PCKL\_V & Float & REB clock rails PCK\_L Voltage \\
SCKU\_V & Float & REB clock rails SCK\_U Voltage \\
SCKL\_V & Float & REB clock rails SCK\_L Voltage \\
RGU\_V & Float & REB clock rails RG\_U Voltage \\
RGL\_V & Float & REB clock rails RG\_L Voltage \\
ODV & Float & REB bias OD voltage \\
OGV & Float & REB bias OG voltage \\
RDV & Float & REB bias RD voltage \\
GDV & Float & REB bias GD voltage \\
ODI & Float & REB bias OD current \\
GDP & Float & REB config bias GD (requested)   \\
RDP & Float & REB config bias RD (requested)   \\
OGP & Float & REB config bias OG (requested)   \\
ODP & Float & REB config bias OD (requested)   \\
CSGATEP & Float & REB config bias CS gate in mA (requested)   \\
SCK\_LOWP & Float & REB DAC SCLK Low (requested) \\
SCK\_HIP & Float & REB DAC SCLK High (requested) \\
PCK\_LOWP & Float & REB DAC PCLK Low (requested) \\
PCK\_HIP & Float & REB DAC PCLK High (requested) \\
RG\_LOWP & Float & REB DAC RG Low (requested) \\
RG\_HIP & Float & REB DAC RG High (requested) \\
AP0\_RC & Integer & CCD ASPIC 0 RC \\
AP1\_RC & Integer & CCD ASPIC 1 RC  \\
AP0\_GAIN & Integer & CCD ASPIC 0 gain \\
AP1\_GAIN & Integer & CCD ASPIC 1 gain \\
AP0\_CLMP & Integer & CCD ASPIC 0 clamp \\
AP1\_CLMP & Integer & CCD ASPIC 1 clamp \\
AP0\_AF1 & Integer & CCD ASPIC 0 af1 \\
AP1\_AF1 & Integer & CCD ASPIC 1 af1 \\
AP0\_TM & Integer & CCD ASPIC 0 Transparent Mode \\
AP1\_TM & Integer & CCD ASPIC 1 Transparent Mode \\
HVBIAS & String & REB HVBias Switch state \\
IDLEFLSH & Integer & IDLE FLUSH setting \\
POWER & Float & Total REB PS power \\
DIGVB & Float & REB PS Digi Voltage before LDO          \\
DIGIB & Float & REB PS Digi Current before LDO          \\
DIGVA & Float & REB PS Digi Voltage after LDO           \\
DIGIA & Float & REB PS Digi Current after LDO         \\
DIGVS & Float & REB PS Digi Voltage after Switch       \\
ANAVB & Float & REB PS Ana Voltage before LDO           \\
ANAIB & Float & REB PS Ana Current before LDO           \\
ANAVA & Float & REB PS Ana Voltage after LDO            \\
ANAIA & Float & REB PS Ana  Current after LDO          \\
ANAIS & Float & REB PS Ana Voltage after Switch        \\
ODVB & Float & REB PS OD Voltage before LDO            \\
ODIB & Float & REB PS OD Current before LDO            \\
ODVA & Float & REB PS OD Voltage after LDO             \\
ODVA2 & Float & REB PS OD Voltage after LDO2            \\
ODIA & Float & REB PS OD Current after LDO             \\
ODVS & Float & REB PS OD Voltage after Switch          \\
CKHVB & Float & REB PS Clk HI Voltage before LDO        \\
CKHIB & Float & REB PS Clk HI Current before LDO        \\
CKHVA & Float & REB PS Clk HI Voltage after LDO         \\
CKHIA & Float & REB PS Clk HI Current after LDO        \\
CKHVS & Float & REB PS Clk HI Voltage after Switch      \\
CKLVB & Float & REB PS Clk LO Voltage before LDO        \\
CKLIB & Float & REB PS Clk LO Current before LDO        \\
CKLVA & Float & REB PS Clk LO Voltage after LDO         \\
CKLV2 & Float & REB PS Clk LO Voltage after LDO2        \\
CKLIA & Float & REB PS Clk LO Current after LDO         \\
CKLVS & Float & REB PS Clk LO Voltage after Switch      \\
HTRVB & Float & REB PS Clk Heater Voltage before LDO    \\
HTRIB & Float & REB PS Clk Heater Current before LDO    \\
HTRVA & Float & REB PS Clk Heater Voltage after LDO     \\
HTRIA & Float & REB PS Clk Heater Current after LDO    \\
HTRVAS & Float & REB PS Clk Heater Voltage after Switch \\
BSSVBS & Float & REB PS HV Bias Voltage before Switch    \\
BSSIBS & Float & REB PS HV Bias Current before Switch    \\
\hline
\end{longtable}



\subsection{Configuration Condition Header}

Header Version: 2


\begin{tabular}{l l l l l}

\hline
Header & Type & Description \\
\hline
EXTNAME & String & Name of the extension \\
FPDEFCS & String & F.P. checksum for default conf. cat. \\
FPDAQCS & String & F.P. checksum for DAQ conf. cat. \\
FPHIDCS & String & F.P. checksum for HardwareId conf. cat. \\
FPINSCS & String & F.P. checksum for Instrument conf. cat. \\
FPLIMCS & String & F.P. checksum for Limits conf. cat. \\
FPRTCCS & String & F.P. checksum for RaftTempControl conf. cat. \\
FPRTCSCS & String & F.P. checksum for RaftTempControlStatus conf. cat. \\
FPRCS & String & F.P. checksum for Rafts conf. cat. \\
FPRLCS & String & F.P. checksum for RaftsLimits conf. cat. \\
FPRPCS & String & F.P. checksum for RaftsRaftsPowerLimits conf. cat. \\
FPSEQCS & String & F.P. checksum for Sequencer conf. cat. \\
FPTIMCS & String & F.P. checksum for timers conf. cat. \\
RPDEFCS & String & REB Power checksum for default conf. cat. \\
RPLIMCS & String & REB Power checksum for Limits conf. cat. \\
RPPOWCS & String & REB Power checksum for Power conf. cat. \\
RPTIMCS & String & REB Power checksum for timers conf. cat. \\
\hline
\end{tabular}



\appendix

\section{Version Changes}

When the FITS header schema is changed in any way there is a corresponding increment to the value stored in the \texttt{HEADVER} FITS card.
This document describes the file format starting with version 2, which is the version that was used from the beginning of commissioning.

Errors in the contents of specific FITS headers are not documented here but are instead handled by the \texttt{astro\_metadata\_translator} package \footnote{\url{https://astro-metadata-translator.lsst.io/}} in conjunction with the \texttt{obs\_lsst} package that are both part of the LSST Science Pipelines \citedsp{PSTN-019}.
In particular, if you are not using the LSST Science Pipelines to read the raw data via the Data Butler \citep{2022SPIE12189E..11J}, some headers are likely to be incorrect or missing if you do not manually run the \texttt{astro\_metadata\_translator.fix\_header()} function on the primary header.

\subsection{Version 3}

\begin{itemize}
\item The \texttt{DATE} header is now correctly being reported in UTC timescale, conforming with the FITS standard \citep{2010A&A...524A..42P}.
\item Removed the \texttt{MJD} header card as it was confusing to users since whilst \texttt{DATE} is part of the standard, the MJD equivalent is not, and it would not be clear whether it was UTC or matched \texttt{TIMESYS}.
\end{itemize}

\section{Acknowledgements}

This material is based upon work supported in part by the National Science Foundation through Cooperative Agreements AST-1258333 and AST-2241526 and Cooperative Support Agreements AST-1202910 and AST-2211468 managed by the Association of Universities for Research in Astronomy (AURA), and the Department of Energy under Contract No.\ DE-AC02-76SF00515 with the SLAC National Accelerator Laboratory managed by Stanford University.
Additional Rubin Observatory funding comes from private donations, grants to universities, and in-kind support from LSST-DA Institutional Members.

% Include all the relevant bib files.
% https://lsst-texmf.lsst.io/lsstdoc.html#bibliographies
\section{References} \label{sec:bib}
\renewcommand{\refname}{} % Suppress default Bibliography section
\bibliography{local,lsst,lsst-dm,refs_ads,refs,books}

% Make sure lsst-texmf/bin/generateAcronyms.py is in your path
\section{Acronyms} \label{sec:acronyms}
\input{acronyms.tex}
% If you want glossary uncomment below -- comment out the two lines above
%\printglossaries





\end{document}
