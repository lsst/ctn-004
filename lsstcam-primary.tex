Header Version: 2

\subsubsection{No Group Assigned}
\begin{tabular}{l l l l l}

\hline
Header & Type & Description \\
\hline
ORIGIN & String & Which site acquired the data \\
BINX & Integer & [pixels] binning along X axis \\
BINY & Integer & [pixels] binning along Y axis \\
CCDGAIN & Float & Rough guess at overall system gain (e-/DNB) \\
CCDNOISE & Float & Rough guess at system noise (e- rms) \\
DETSIZE & String &  \\
\hline
\end{tabular}


\subsubsection{Date, night and basic image information}
\begin{tabular}{l l l l l}

\hline
Header & Type & Description \\
\hline
DATE & Date & Creation Date and Time of File \\
MJD & MJD & Modified Julian Date that the file was written \\
IMGTYPE & String & BIAS, DARK, FE55, FLAT, FLAT\_<lam>, SPOT, PPUMP \\
DATE-OBS & Date & Time at the start of integration \\
MJD-OBS & MJD & Modified Julian Date derived from DATE-OBS \\
DATE-TRG & Date & TAI Date of the image trigger (readout) \\
MJD-TRG & MJD & Modified Julian Date of image trigger \\
OBSID & String & The image name or obs-id \\
DATE-BEG & Date & Time at the start of integration \\
MJD-BEG & MJD & Modified Julian Date derived from DATE-BEG \\
DATE-END & Date & End date of the observation \\
MJD-END & MJD & Date derived from DATE-END \\
BUNIT & String & Brightness units for pixel array \\
TIMESYS & String & The time scale used \\
GROUPID & String &  \\
\hline
\end{tabular}


\subsubsection{Telescope info, location, observer}
\begin{tabular}{l l l l l}

\hline
Header & Type & Description \\
\hline
INSTRUME & String & Instrument \\
TELESCOP & String & Telescope \\
OBS-LONG & String & [deg] Observatory east longitude \\
OBS-LAT & String & [deg] Observatory latitude \\
OBS-ELEV & String & [m] Observatory elevation \\
OBSGEO-X & String & [m] X-axis Geocentric coordinate \\
OBSGEO-Y & String & [m] Y-axis Geocentric coordinate \\
OBSGEO-Z & String & [m] Z-axis Geocentric coordinate \\
\hline
\end{tabular}


\subsubsection{Pointing info, etc.}
\begin{tabular}{l l l l l}

\hline
Header & Type & Description \\
\hline
RA & String & RA commanded from pointing component \\
DEC & String & DEC commanded from pointing component \\
RASTART & String & RA of telescope from AZSTART and ELSTART \\
DECSTART & String & DEC of telescope from AZSTART and ELSTART \\
RAEND & String & RA of telescope from AZEND and ELEND \\
DECEND & String & DEC of telescope from AZEND and ELEND \\
ROTPA & String & Rotation angle relative to the sky (deg) \\
ROTCOORD & String & Telescope Rotation Coordinates \\
HASTART & String & [HH:MM:SS] Telescope hour angle at start \\
ELSTART & String & [deg] Telescope zenith distance at start \\
AZSTART & String & [deg] Telescope azimuth angle at start \\
AMSTART & String & Airmass at start \\
HAEND & String & [HH:MM:SS] Telescope hour angle at end \\
ELEND & String & [deg] Telescope zenith distance at end \\
AZEND & String & [deg] Telescope azimuth angle at end \\
AMEND & String & Airmass at end \\
TRACKSYS & String & Tracking system RADEC, AZEL, PLANET, EPHEM \\
RADESYS & String & Equatorial coordinate system FK5 or ICRS \\
FOCUSZ & String & Focus Z position \\
OBJECT & String & Name of the observed object \\
VIGNETTE & String & Is the telescope vignetted? \\
VIGN\_MIN & String & Minimum vignetting? \\
\hline
\end{tabular}


\subsubsection{Image-identifying used to build OBS-ID}
\begin{tabular}{l l l l l}

\hline
Header & Type & Description \\
\hline
TESTTYPE & String & BIAS, DARK, FE55, FLAT, LAMBDA, PERSISTENCE, SPOT, SFLAT\_<lam>, TRAP \\
CAMCODE & String & The "code" for AuxTel | ComCam | Main Camera \\
CONTRLLR & String & The controller (e.g. O for OCS, C for CCS) \\
DAYOBS & String & The observation day as defined in the image name  \\
SEQNUM & Integer & The sequence number from the image name \\
EMUIMAGE & String & Image being emulated (from 2-day store) \\
\hline
\end{tabular}


\subsubsection{Additional Keys Information from Camera}
\begin{tabular}{l l l l l}

\hline
Header & Type & Description \\
\hline
PROGRAM & String & Name of the program \\
REASON & String & Reason for observation \\
\hline
\end{tabular}


\subsubsection{Image sequence numbers}
\begin{tabular}{l l l l l}

\hline
Header & Type & Description \\
\hline
CURINDEX & Integer & Index number for exposure within the sequence \\
MAXINDEX & Integer & Number of requested images in sequence \\
\hline
\end{tabular}


\subsubsection{Test Stand information}
\begin{tabular}{l l l l l}

\hline
Header & Type & Description \\
\hline
TSTAND & String & Test Stand/Physical location \\
\hline
\end{tabular}


\subsubsection{Information from Camera (Common block)}
\begin{tabular}{l l l l l}

\hline
Header & Type & Description \\
\hline
IMAGETAG & String & DAQ Image id (Hex) \\
OBSANNOT & String & DAQ image annotation \\
\hline
\end{tabular}


\subsubsection{Information from Camera}
\begin{tabular}{l l l l l}

\hline
Header & Type & Description \\
\hline
RUNNUM & String & The Run Number \\
CONTROLL & String & Duplicates INSTRUME \\
CCD\_MANU & String & CCD Manufacturer \\
CCD\_TYPE & String & CCD Model Number \\
TEMP\_SET & Float & Temperature set point (deg C) \\
CCDSLOT & String & The CCD Slot \\
RAFTBAY & String & The RAFT Bay \\
FIRMWARE & String & DAQ firmware version (Hex) \\
PLATFORM & String & DAQ platform version \\
CONTNUM & String & REB serial \# (Hex) \\
DAQVERS & String & DAQ version \\
DAQPART & String & DAQ partition \\
DAQFOLD & String & DAQ folder the image was initially created in \\
SEQFILE & String & Sequencer file name \\
SEQNAME & String & Sequencer file name \\
SEQCKSUM & String & Checksum of Sequencer \\
LSST\_NUM & String & LSST Assigned CCD Number \\
CCD\_SERN & String & Manufacturers’ CCD Serial Number \\
REBNAME & String & LSST Assigned Name REB name \\
RAFTNAME & String & LSST Assigned Raft name \\
FPVERS & String & The focal-plane version number \\
IHVERS & String & The image-handling version number \\
\hline
\end{tabular}


\subsubsection{Filter/grating information}
\begin{tabular}{l l l l l}

\hline
Header & Type & Description \\
\hline
FILTBAND & String & Name of the filter band \\
FILTER & String & Name of the physical filter \\
FILTPOS & String & Filter measured position of filter \\
FILTSLOT & String & Filter home slot \\
\hline
\end{tabular}


\subsubsection{Exposure-related information}
\begin{tabular}{l l l l l}

\hline
Header & Type & Description \\
\hline
EXPTIME & Float & Exposure Time in Seconds \\
DARKTIME & Float & Dark Time in Seconds (see TSEIA-91) \\
SHUTTIME & String & Measured shutter exposure time \\
\hline
\end{tabular}


\subsubsection{Weather information}
\begin{tabular}{l l l l l}

\hline
Header & Type & Description \\
\hline
AIRTEMP & String & Position of slide \\
PRESSURE & String & [Pascals] Atmospheric Pressure \\
HUMIDITY & String & Percentage of Relative Humidity \\
WINDSPD & String & [m/s] Wind Speed \\
WINDDIR & String & [deg] Wind Direction \\
SEEING & String & [arcsec] Seeing from FWHM DIMM measurement \\
\hline
\end{tabular}


\subsubsection{Header information}
\begin{tabular}{l l l l l}

\hline
Header & Type & Description \\
\hline
HEADVER & Integer & Version number of header \\
FILENAME & String & Original name of the file \\
\hline
\end{tabular}


\subsubsection{Stuttered Image Support}
\begin{tabular}{l l l l l}

\hline
Header & Type & Description \\
\hline
HIERARCH STUTTER ROWS & Integer & The number of row shifts per stutter \\
HIERARCH STUTTER DELAY & Float & [s] The delay between stutters \\
HIERARCH STUTTER NSHIFTS & Integer & The number of stutters \\
\hline
\end{tabular}

